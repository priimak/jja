\documentclass[40pt,letterpaper]{article}
\usepackage[utf8]{inputenc}
\usepackage{graphicx}
\usepackage{amsmath}
\usepackage{amsfonts}
\usepackage{amssymb}
\usepackage{graphicx}
\usepackage{cancel}
\usepackage{datetime}
\usepackage{fullpage}
\DeclareGraphicsExtensions{.pdf,.png,.jpg} 
\author{Dmitri Priimak}
\title{Josephson Junctions Array}
\def\spvec#1{\left(\vcenter{\halign{\hfil$##$\hfil\cr \spvecA#1;;}}\right)}
\def\spvecA#1;{\if;#1;\else #1\cr \expandafter \spvecA \fi}
\begin{document}
 \begin{center}
  \underline{Array of Josephson Junctions in the periodic a/c field.}
 \end{center}
  \begin{center}
    \underline{Rev. 1}
  \end{center}
   	\section{Many Josephson Junctions}
  	\begin{eqnarray}
  	\frac{\text{d}q}{\text{d}t} &=& q_1 \\
  	\frac{\text{d}\phi_k}{\text{d}t} &=& i_{dc} 
  		+ i_1 \cos(\Omega_{1}t) - \sin(\phi_k) 
  		- q_1 \\
  	\frac{\text{d}q_1}{\text{d}t} &=&
  	\frac{1}{l}\left[N \left( i_{dc} 
  		+ i_1 \cos(\Omega_1 t) - q_1\right)
  		- \sum_{k=0}^{N-1} \sin(\phi_k) 
  		\right]
  		-\frac{r}{l}q_1 - \Omega_0^2 q
  	\end{eqnarray}
  	Let us introduce new vector $\mathbf{r} = \spvec{q;\{\phi_k\};q_1}$ and a vector function $\mathbf{f}$ corresponding to the functions on
  	the right of the equal sign in the equations above.
	\begin{equation}
  		\mathbf{f}(t, \mathbf{r})=\spvec{q;
  		i_{dc} 
  		  		+ i_1 \cos(\Omega_{1}t) - \sin(\phi_k) 
  		  		- q_1;
  		\frac{1}{l}\left[N \left( i_{dc} 
  		  		+ i_1 \cos(\Omega_1 t) - q_1\right)
  		  		- \sum_{k=0}^{N-1} \sin(\phi_k) 
  		  		\right]
  		  		-\frac{r}{l}q_1 - \Omega_0^2 q}
  	\end{equation}
  	Then set of differenetial equations can be written simply as:
  	\begin{equation}
  	\frac{\text{d}\mathbf{r}}{\text{d}t}=\mathbf{f}(t, \mathbf{r})
  	\end{equation}
  	We use forth order Runge-Kutta method to solve this equation. To than end we step over time with step $\Delta t$. And introduce time step supperscript to $\mathbf{r}$. Thus $\mathbf{r}^t$ and $\mathbf{r}^{t+1}$ are values of $\mathbf{r}$ for two consequetive time steps.
  	\begin{eqnarray}
  	\mathbf{k}_1 &=& \mathbf{f}(t, \mathbf{r}^t) \\
  	\mathbf{\tilde{r}} &=& \mathbf{r}^t + \frac{\Delta t}{2}\mathbf{k}_1 \\
  	\mathbf{k}_2 &=& \mathbf{f} \left(t + \frac{\Delta t}{2}, \mathbf{\tilde{r}}\right) \\
  	\mathbf{\tilde{r}} &=& \mathbf{r}^t + \frac{\Delta t}{2}\mathbf{k}_2 \\
  	\mathbf{k}_3 &=& \mathbf{f} \left(t + \frac{\Delta t}{2}, \mathbf{\tilde{r}}\right) \\  	
  	\mathbf{\tilde{r}} &=& \mathbf{r}^t + \Delta t\mathbf{k}_3 \\
  	\mathbf{k}_4 &=& \mathbf{f} \left(t + \Delta t, \mathbf{\tilde{r}}\right)	
  	\end{eqnarray}
  	\begin{equation}
  	\mathbf{r}^{t+1} = \mathbf{r}^t + \frac{\Delta t}{6}\left(\mathbf{k}_1 + 2 \mathbf{k}_2 + 2 \mathbf{k}_3 + \mathbf{k}_4\right)
  	\end{equation}
  	\newpage
  	\section{One Josephson Junction}
  	\begin{eqnarray}
  	\frac{\text{d}q}{\text{d}t} &=& q_1 \\
  	\frac{\text{d}\phi}{\text{d}t} &=& i_{dc} 
  		+ i_1 \cos(\Omega_{1}t) - \sin(\phi) 
  		- q_1 \\
  	\frac{\text{d}q_1}{\text{d}t} &=&
	\frac{i_{dc}}{l} + 
	\frac{i_1}{l}\cos(\Omega_1 t) 
	- \frac{1 + r}{l}q_1 - \frac{1}{l}\sin(\phi) 
	- \Omega_0^2 q
  	\end{eqnarray}
  	Free parameters $i_{dc}, i_1, r, l, \Omega_0, \Omega_1$
  	\subsection{First Order Euler}
  	\begin{eqnarray}
  	q^{t+\Delta t} &=& q^t + q_1 \Delta t \\
  	\phi^{t+\Delta t} &=& \phi^t + \Delta t
  	\left[
  	i_{dc} + i_1 \cos(\Omega_{1}t) - \sin(\phi) 
  	- q^{t}_1
  	\right] \\
  	q^{t+\Delta t}_1 &=& q^t_1 + \Delta t
  	\left[
	\frac{i_{dc}}{l} + 
	\frac{i_1}{l}\cos(\Omega_1 t) 
	- \frac{1 + r}{l}q^t_1 - \frac{1}{l}\sin(\phi^t) 
	- \Omega_0^2 q^t  	
  	\right]
  	\end{eqnarray}
  	
\end{document}
